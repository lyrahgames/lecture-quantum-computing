\documentclass[9pt]{article}
\usepackage{extsizes}
\usepackage[utf8]{inputenc}
\usepackage[T1]{fontenc}
\usepackage[a4paper,left=20mm,top=20mm,right=20mm,bottom=20mm]{geometry}
\pagestyle{empty}
\usepackage[light,sfdefault]{roboto}
\usepackage{multicol}

\begin{document}
  \begin{multicols}{2}
    \begin{itemize}
      \item{
        Vorbemerkungen
        \begin{itemize}
          \item Leistungsfähigkeit von Rechnern
          \item Problemklassen
        \end{itemize}
      }
      \item{
        Boolesche Algebra
        \begin{itemize}
          \item Rechnen im $\mathbf{F}_2$
          \item 1-Bit-Gatter
          \item 2-Bit-Gatter
          \item $n$-Bit-Gatter
        \end{itemize}
      }
      \item Reversible Operationen
        \begin{itemize}
          \item Definition: reversible Gatter
          \item Landauer Prinzip
          \item Shannon Entropie
          \item 2-Niveau-Systeme und Bitverluste
        \end{itemize}
      \item Konstruktion universeller und reversibler Gatter
        \begin{itemize}
          \item 1-1-Gatter: Identität, NOT
          \item 2-2-Gatter: linear, COPY, SWAP, CNOT
          \item 3-3-Gatter: nichtlinear, Toffoli
        \end{itemize}
      \item Quantentheorie
        \begin{itemize}
          \item Hilbert-Raum
          \item Operatoren
          \item ONB, inneres Produkt, äußeres Produkt
          \item Erwartungswert
          \item orthogonaler Projektor
          \item Normaloperator
          \item unitärer Basiswechsel, biorthogonale Entwicklung
          \item inneres Produkt nach Hilbert-Schmidt
          \item Superoperator
          \item Polarzerlegung, Spektral-Theorem
        \end{itemize}
      \item Qubits
        \begin{itemize}
          \item Realisierung
          \item Definition
          \item fundamentale Operatoren und Pauli-Matrizen
          \item Spin-Hamilton-Operator
          \item Interpretation von 1-Bit-Gattern
          \item Hadamard-Operator und Quantenparallelismus
        \end{itemize}
      \item Systeme aus zwei Qubits
        \begin{itemize}
          \item ohne Wechselwirkung
          \item mit Wechselwirkung
          \item Verschränkung
          \item Maß der Verschränkung, Konkurrenz
          \item Bell-Zustände
          \item Dichteoperator und Eigenschaften
          \item inkohärente Überlagerung
          \item reiner Zustand
          \item Reinheitsmaß
        \end{itemize}
      \item Bloch-Kugel
        \begin{itemize}
          \item allgemeiner Dichteoperator
          \item Eigenwert-Problem
          \item Bloch-Vektor auf Bloch-Sphäre
          \item Reinheitsgrad-Konkurrenz-Beziehung
        \end{itemize}
      \item No-cloning Theorem
        \begin{itemize}
          \item formale Definition: Quantenkopierer
          \item Widerspruchsbeweis
        \end{itemize}
      \item Quantengatter mit Einzel-Qubit
        \begin{itemize}
          \item allgemeine Phase
          \item $\pi/8$-Gatter
          \item Verallgemeinerung von NOT
          \item Quanten-Euler-Formel
          \item Rotationsverhalten und allgemeinste Drehung
          \item allgemeiner unitärer Operator
          \item ABC-Operatoren
          \item Polardarstellung mit Rotation
        \end{itemize}
      \item 2-Qubit-Systeme
        \begin{itemize}
          \item Verschränkungsmaß und Bell-Zustände
          \item partielle Spurbildung
          \item Wurzel aus $\sigma_\mathrm{X}$
        \end{itemize}
      \item 2-Qubit-Gatter
        \begin{itemize}
          \item CNOT
          \item Verallgemeinerung: CU
          \item Toffoli
        \end{itemize}
      \item Universeller Satz von Quantengattern
        \begin{itemize}
          \item CNOT mit Verallgemeinerung
          \item $\pi/8$-Gatter, Hadamard-Gatter
          \item Universalsätze
          \item 3-Qubit-Systeme
          \item Approximation von Einzel-Qubit-Gattern mit erster und zweiter Rotation
        \end{itemize}
      \item Quanten-Algorithmen
        \begin{itemize}
          \item grundlegender Ablauf und Bedingungen
          \item Deutsch-Algorithmus
          \item Quanten-Fourier-Transformation mit Schaltung
          \item Periodensuche und Anwendung QFT mit Euklid-Algorithmus
          \item Primfaktorzerlegung durch Ordnungsbestimmung
          \item Modulare Exponentiation
          \item Shor-Algorithmus
          \item Suchalgorithmus nach Grover
        \end{itemize}
      \item Fehlertheorie
        \begin{itemize}
          \item Klassische Fehlerkorrektur
          \item Bit-Swap und Parität
          \item Shor-Code
          \item Steame-Code
          \item optimierter Code
          \item etwas mehr
        \end{itemize}
    \end{itemize}
  \end{multicols}
\end{document}